\documentclass[12pt,a4paper]{article}

% ----------------------
% 基础文档配置
% ----------------------
\usepackage{ctex}                 % 中文支持核心包
\usepackage{fontspec}             % 字体设置扩展包
\usepackage{geometry}             % 页面布局设置
\usepackage{setspace}             % 行距设置
\usepackage{graphicx}             % 图片插入
\usepackage{caption}              % 图表标题设置
\usepackage{subcaption}           % 子图表支持
\usepackage{float}                % 图表浮动控制
\usepackage{amsmath,amssymb}      % 数学公式支持
\usepackage{fancyhdr}             % 页眉页脚定制
\usepackage{tocloft}              % 目录格式设置
\usepackage{natbib}               % 参考文献引用
\usepackage{titlesec}             % 章节标题格式设置
\usepackage{etoolbox}             % 工具包(备用)

% ----------------------
% 字体全局设置
% ----------------------
\setmainfont{Times New Roman}     % 英文主字体
\setsansfont{Arial}               % 无衬线英文字体
\setmonofont{Courier New}         % 等宽英文字体
\newfontfamily\SimSun{SimSun}       % 定义宋体命令
\newfontfamily\KaiTi{KaiTi}         % 定义楷体命令

% ----------------------
% 页面布局设置
% ----------------------
\geometry{
  left=2.5cm,
  right=2.5cm,
  top=2.5cm,
  bottom=2.5cm,
  bindingoffset=0.5cm
}
\linespread{1.25}                  % 1.25倍行距

% ----------------------
% 页眉页脚设置
% ----------------------
\pagestyle{fancy}
\fancyhf{}
\fancyfoot[C]{\thepage}

% ----------------------
% 目录格式设置
% ----------------------
\renewcommand{\cftsecleader}{\cftdotfill{\cftdotsep}}
\renewcommand{\contentsname}{目录} % 目录标题汉化

% ----------------------
% 章节标题格式
% ----------------------
\titleformat{\section}{\zihao{-4}\bfseries}{\thesection}{1em}{}
\titleformat{\subsection}{\zihao{-4}}{\thesubsection}{1em}{}

% ----------------------
% 图表标题格式
% ----------------------
\captionsetup{
  font={small},
  labelfont=bf,
  labelsep=period
}

% ----------------------
% 文档开始
% ----------------------
\begin{document}

% ----------------------
% 封面页(严格按格式要求)
% ----------------------
\begin{titlepage}
\centering

% 珠海科技学院(宋体,初号,居中;段前空小四号一行,段后空初号一行)
\vspace{\baselineskip}            % 段前空小四号一行(12pt)
{\SimSun\zihao{0} 珠海科技学院 \par}
\vspace{\baselineskip}            % 段后空初号一行(14.85pt)


% 毕业论文(宋体,初号,加粗,居中,字间空一格)
\vspace{1cm}                     % 额外间隔
{\SimSun\zihao{0}\bfseries 毕\ 业\ 论\ 文 \par}
\vspace{1cm}                     % 段后间隔

% 论文标题(楷体,小一,加粗,居中;段前空初号一行,段后空小一四行)
\vspace{\baselineskip}            % 段前空初号一行
{\kaishu\zihao{-1}\bfseries 基于微服务的健康管理平台的研究与设计 \par}
\vspace{4\baselineskip}           % 段后空小一四行

% 作者信息(左对齐,小四字体)
\begin{flushleft}
{\zihao{4}
学院:XXX学院 \par
专业名称:XXX \par
学生姓名:XXX \par
学号:XXXXXXXXX \par
指导教师姓名职称:XXX XXX \par
完成日期:2025年6月 \par
}
\end{flushleft}

\vfill
(本页请勿编页码)
\end{titlepage}

% ----------------------
% 摘要部分
% ----------------------
\section*{摘\quad 要}
\addcontentsline{toc}{section}{摘要}

本文针对微服务XXXXXXXXXXXX\\XXXXXXXXXXXXXXXXXXXXXXXXXXXXXX,
提出并实现了一个健康管理平台。XXXXXXXXXXXXXXXXXXXXXXXX。

关键词:健康管理;微服务;Spring Boot;Spring Cloud;MySQL


% ----------------------
% 英文摘要部分
% ----------------------
\section*{Abstract}
\addcontentsline{toc}{section}{Abstract}

This paper presents a study on health management systems based on microservices architecture. The system uses Spring Boot and MySQL as core components. The result shows significant performance improvements…

Key words: Health Management; Microservices; Spring Boot; MySQL


% ----------------------
% 目录生成
% ----------------------
\clearpage
\pagenumbering{Roman}
\tableofcontents
\clearpage
\pagenumbering{arabic}

% ----------------------
% 正文章节(示例引用)
% ----------------------
\section{绪论}

随着医疗信息化的不断推进,传统的单体健康系统在功能拓展、系统维护及用户体验方面逐渐暴露出诸多问题。因此,采用基于微服务架构的健康管理平台,具有高度灵活性、可维护性以及良好的用户交互性。

\subsection{研究背景及意义}
\subsubsection{选题背景}
近年来,随着居民健康意识的增强和互联网技术的发展,医疗健康管理成为大众关注的热点。与此同时,传统医疗系统的单体架构已难以满足复杂业务需求,故需要引入分布式架构进行系统重构。

\subsubsection{研究意义}
该平台采用微服务架构可大幅度提升系统模块间的解耦性,提高系统的可维护性和可靠性。理论上丰富了微服务在医疗领域的研究实践,实际中也为类似场景提供借鉴。

\subsection{研究内容和方法}
本论文研究内容包括平台架构设计、模块实现与部署测试,主要采用需求分析、系统设计、原型实现、测试验证等方法。

\subsection{关于论文的一些补充说明}
本平台为机构委托开发,需考虑数据安全、性能优化和未来可扩展性等因素。

\subsection{论文章节安排}
第2章介绍关键技术;第3章进行系统需求分析;第4章为系统设计与实现;第5章讲述部署与测试;第6章为总结与展望。

% \section{关键技术研究}

本课题研究的主要内容是设计并实现基于分布式微服务架构的健康管理平台。为满足平台的高可用性、可扩展性、模块解耦等需求,本文采用了一系列关键技术,分为架构层面与实现层面两个方面。

\subsection{Spring Cloud 分布式微服务架构}

Spring Cloud 是一套基于 Spring Boot 的微服务开发工具集,提供了服务注册与发现、配置管理、断路器、智能路由、微代理等多种功能,简化了分布式系统中各模块的协作问题。

在本平台中,使用 Eureka 实现服务注册与发现;使用 Spring Cloud Gateway 实现网关路由与请求转发;结合 Spring Config Server 统一配置管理,实现了平台服务的模块化、可独立部署与维护。

\subsection{Spring Boot 应用框架}

Spring Boot 是构建独立、生产级 Spring 应用的快速开发框架,具有自动配置、起步依赖、嵌入式服务器支持等优点。

在本平台中,所有服务模块均基于 Spring Boot 构建,简化了项目结构与依赖配置。开发人员只需关注业务逻辑本身,极大提升了开发效率与代码可维护性。

\subsection{Redis 缓存技术}

Redis 是一款高性能的内存键值数据库,支持多种数据结构。它在微服务系统中常用于缓存热点数据、分布式锁、会话共享等场景。

平台中的用户会话数据、频繁访问的健康资讯等内容均通过 Redis 进行缓存,显著降低数据库访问压力,提高平台整体响应性能。

\subsection{Elasticsearch 搜索引擎}

为提升平台内容检索的效率与智能化程度,引入 Elasticsearch 实现全文搜索服务。其支持分布式、实时、高可用的数据索引和查询能力。

用户可通过关键词快速查询健康知识、平台内容、历史记录等信息,大幅优化了用户体验。

\subsection{Docker 容器化部署}

为方便平台部署、测试与扩展,平台采用 Docker 进行容器化部署。每个微服务独立打包为镜像,通过 Docker Compose 管理服务编排,保证系统环境一致性和可复现性。

\subsection{本章小结}

本章对平台采用的关键技术进行了详细阐述。这些技术从架构支撑、应用框架、缓存加速、搜索优化到容器化部署,全面保障了系统在复杂业务环境中的性能、稳定性与可维护性。

% \section{需求分析}

需求分析是软件系统开发中的关键环节,关系到系统的功能实现与用户体验。本章将从功能需求、非功能性需求以及数据需求三个方面对健康管理平台进行详细分析。

\subsection{功能需求}

平台的主要功能面向两类用户:普通用户与平台管理员。核心功能包括健康档案管理、在线问诊、文章资讯展示、问卷测评与后台管理等模块。

\subsubsection{健康档案模块}
- 用户可查看、编辑个人健康信息;
- 支持体检报告上传与数据可视化展示;
- 系统对接健康设备数据(如步数、血压等)。

\subsubsection{在线问诊模块}
- 用户可根据症状选择科室与医生;
- 提供图文问诊与挂号预约服务;
- 支持订单生成、支付、记录追踪。

\subsubsection{资讯推荐模块}
- 系统展示健康相关资讯;
- 基于用户行为提供个性化推荐;
- 管理员可发布与编辑内容。

\subsubsection{问卷与评估模块}
- 提供心理/身体健康测评问卷;
- 自动计算结果并给出建议;
- 用户可查看历史评测记录。

\subsubsection{后台管理模块}
- 支持用户、文章、问诊订单的管理;
- 提供系统数据统计与图表展示;
- 管理员权限分级控制。

\subsection{非功能性需求}

\subsubsection{性能需求}
- 支持并发用户量不少于 1000;
- 关键请求接口响应时间不超过 500ms;
- 系统日处理数据量达到 10 万条以上。

\subsubsection{安全需求}
- 用户数据需加密存储,符合数据安全规范;
- 所有接口采用 HTTPS 加密传输;
- 平台需具备身份认证与访问控制机制。

\subsubsection{可用性与可维护性}
- 系统年可用性需达到 99.9\%;
- 服务需支持水平扩展与灰度部署;
- 各模块应独立解耦,便于后续维护与升级。

\subsection{数据需求}

\subsubsection{数据实体}
主要实体包括用户、医生、健康档案、订单、文章、问卷等,需构建统一的数据模型。

\subsubsection{数据存储与访问}
平台采用 MySQL 存储核心业务数据,Redis 进行缓存加速;Elasticsearch 用于检索相关信息。

\subsection{本章小结}

本章从功能、非功能性以及数据需求三个方面对平台进行了详尽分析,为后续的系统设计与实现提供了明确依据与指导方向。

% \section{设计与实现}

本章将介绍健康管理平台的系统设计与实现过程,包括系统架构设计、数据库设计、主要功能模块的实现等内容。该平台采用分布式微服务架构,从整体到局部进行了分层设计,以实现系统的高内聚、低耦合、高扩展性。

\subsection{系统总体架构设计}

平台采用三层结构:接入层、业务层、数据层。各层之间职责清晰,协同工作如图~\ref{fig:arch} 所示。

\begin{figure}[H]
\centering
\includegraphics[width=0.8\textwidth]{fig/arch.pdf}
\caption{平台整体架构图}
\label{fig:arch}
\end{figure}

\begin{itemize}
  \item 接入层:主要包括网关(Gateway)和认证服务,用于统一入口控制、用户认证与权限管理;
  \item 业务层:封装具体业务模块,如用户服务、问诊服务、内容服务等;
  \item 数据层:包括数据库服务、缓存服务(Redis)、文件服务等。
\end{itemize}

\subsection{数据库设计}

平台核心数据通过 MySQL 存储,缓存使用 Redis,搜索引擎使用 Elasticsearch。以“用户表”为例,其逻辑结构如表~\ref{tab:user} 所示。

\begin{table}[H]
\centering
\caption{用户表结构设计}
\label{tab:user}
\begin{tabular}{|c|c|c|c|l|}
\hline
字段名 & 类型 & 长度 & 主键 & 说明 \\
\hline
user\_id & BIGINT & 20 & 是 & 用户唯一标识 \\
username & VARCHAR & 50 & 否 & 用户名 \\
password & VARCHAR & 100 & 否 & 加密后的密码 \\
email & VARCHAR & 100 & 否 & 邮箱地址 \\
created\_at & DATETIME & - & 否 & 注册时间 \\
\hline
\end{tabular}
\end{table}

\subsection{主要模块实现}

\subsubsection{用户管理模块}

该模块提供用户注册、登录、信息修改、身份验证等功能,支持基于 JWT 的 Token 身份认证机制。密码使用 Bcrypt 加密存储。

\subsubsection{在线问诊模块}

该模块实现用户与医生的图文问诊交流,采用 WebSocket 实时通信方式。后端管理用户问诊订单的生成、处理与状态追踪,并支持历史问诊记录查询。

\subsubsection{内容服务模块}

管理员可通过后台发布健康文章资讯,用户可浏览并收藏文章。系统支持关键词搜索,并通过 Elasticsearch 实现模糊匹配和排序推荐。

\subsubsection{问卷量表模块}

支持用户填写心理/健康自评问卷,系统根据用户回答计算分数并给出反馈建议。支持历史评测记录的展示与数据可视化。

\subsection{接口示例设计}

以下为一个用户登录接口的示例说明:

\begin{itemize}
  \item URL:`POST /api/auth/login`
  \item 参数:`username`, `password`
  \item 返回:用户基本信息 + JWT Token
  \item 错误码:401(认证失败)
\end{itemize}

\subsection{本章小结}

本章详细介绍了平台的系统架构、数据库设计、核心模块的功能与实现方法。通过模块化与服务化设计,平台具备良好的扩展性和维护性,为后续部署与测试提供了基础。

% \section{部署与测试}

本章主要介绍健康管理平台的部署方式、所需软硬件环境及主要功能测试方案,验证系统的稳定性、可靠性与实际使用性能。

\subsection{部署环境}

\subsubsection{软硬件环境}

平台推荐部署于 Linux 服务器(如 Ubuntu 20.04),所需环境如表~\ref{tab:env} 所示:

\begin{table}[H]
\centering
\caption{部署环境配置}
\label{tab:env}
\begin{tabular}{|c|c|}
\hline
组件 & 配置说明 \\
\hline
操作系统 & Ubuntu 20.04 LTS \\
Java 环境 & JDK 17 \\
数据库 & MySQL 8.0 \\
缓存 & Redis 7 \\
容器管理 & Docker 24 + Docker Compose \\
搜索引擎 & Elasticsearch 8.10 \\
\hline
\end{tabular}
\end{table}

\subsubsection{服务部署流程}

1. 使用 Git 克隆后端项目代码;
2. 通过 Docker Compose 启动微服务容器;
3. Nginx 配置反向代理;
4. MySQL 初始化数据库结构;
5. 运行测试脚本,确认服务启动正常。

\subsection{功能测试}

\subsubsection{测试方法}

采用黑盒测试方式,设计测试用例覆盖各模块主要功能点,包括:

\begin{itemize}
  \item 注册登录测试;
  \item 问诊功能流程测试;
  \item 文章推荐与检索测试;
  \item 管理后台功能权限测试;
  \item 系统异常与边界条件处理。
\end{itemize}

\subsubsection{注册功能测试用例(示例)}

\begin{itemize}
  \item 测试编号:U1-1-1;
  \item 输入:合法用户名、密码、邮箱;
  \item 期望结果:注册成功,返回用户信息;
  \item 异常输入测试:用户名为空、密码太短等应返回错误提示;
  \item 实际结果:全部通过,符合预期。
\end{itemize}

\subsection{性能测试}

使用 Apache JMeter 对接口并发访问性能进行压测,结果如下:

- 并发 500 用户时,登录接口平均响应时间:312ms;
- 并发 1000 用户时,系统仍可稳定响应,CPU 占用维持在 70\% 以下。

\subsection{本章小结}

本章对平台的部署流程、运行环境及功能测试过程进行了说明。经测试验证,系统在功能完整性、稳定性与性能表现方面均满足设计预期。


% ----------------------
% 参考文献部分(暂时注释下面三行)
% ----------------------
% \clearpage
% \bibliographystyle{unsrt}
% \bibliography{ref}

% ----------------------
% 致谢部分
% ----------------------
\clearpage
\section*{致谢}
\addcontentsline{toc}{section}{致谢}
感谢指导老师 XXX 的悉心指导,感谢家人的支持与理解。在论文撰写过程中,还得到了同学们的热心帮助,在此一并表示诚挚的谢意。

% ----------------------
% 附录部分(可选)
% ----------------------
% \clearpage
% \appendix
% \input{sections/appendix.tex}

\end{document}
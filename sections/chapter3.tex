\section{需求分析}

需求分析是软件系统开发中的关键环节,关系到系统的功能实现与用户体验。本章将从功能需求、非功能性需求以及数据需求三个方面对健康管理平台进行详细分析。

\subsection{功能需求}

平台的主要功能面向两类用户:普通用户与平台管理员。核心功能包括健康档案管理、在线问诊、文章资讯展示、问卷测评与后台管理等模块。

\subsubsection{健康档案模块}
- 用户可查看、编辑个人健康信息;
- 支持体检报告上传与数据可视化展示;
- 系统对接健康设备数据(如步数、血压等)。

\subsubsection{在线问诊模块}
- 用户可根据症状选择科室与医生;
- 提供图文问诊与挂号预约服务;
- 支持订单生成、支付、记录追踪。

\subsubsection{资讯推荐模块}
- 系统展示健康相关资讯;
- 基于用户行为提供个性化推荐;
- 管理员可发布与编辑内容。

\subsubsection{问卷与评估模块}
- 提供心理/身体健康测评问卷;
- 自动计算结果并给出建议;
- 用户可查看历史评测记录。

\subsubsection{后台管理模块}
- 支持用户、文章、问诊订单的管理;
- 提供系统数据统计与图表展示;
- 管理员权限分级控制。

\subsection{非功能性需求}

\subsubsection{性能需求}
- 支持并发用户量不少于 1000;
- 关键请求接口响应时间不超过 500ms;
- 系统日处理数据量达到 10 万条以上。

\subsubsection{安全需求}
- 用户数据需加密存储,符合数据安全规范;
- 所有接口采用 HTTPS 加密传输;
- 平台需具备身份认证与访问控制机制。

\subsubsection{可用性与可维护性}
- 系统年可用性需达到 99.9\%;
- 服务需支持水平扩展与灰度部署;
- 各模块应独立解耦,便于后续维护与升级。

\subsection{数据需求}

\subsubsection{数据实体}
主要实体包括用户、医生、健康档案、订单、文章、问卷等,需构建统一的数据模型。

\subsubsection{数据存储与访问}
平台采用 MySQL 存储核心业务数据,Redis 进行缓存加速;Elasticsearch 用于检索相关信息。

\subsection{本章小结}

本章从功能、非功能性以及数据需求三个方面对平台进行了详尽分析,为后续的系统设计与实现提供了明确依据与指导方向。

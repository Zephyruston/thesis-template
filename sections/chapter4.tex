\section{设计与实现}

本章将介绍健康管理平台的系统设计与实现过程,包括系统架构设计、数据库设计、主要功能模块的实现等内容。该平台采用分布式微服务架构,从整体到局部进行了分层设计,以实现系统的高内聚、低耦合、高扩展性。

\subsection{系统总体架构设计}

平台采用三层结构:接入层、业务层、数据层。各层之间职责清晰,协同工作如图~\ref{fig:arch} 所示。

\begin{figure}[H]
\centering
\includegraphics[width=0.8\textwidth]{fig/arch.pdf}
\caption{平台整体架构图}
\label{fig:arch}
\end{figure}

\begin{itemize}
  \item 接入层:主要包括网关(Gateway)和认证服务,用于统一入口控制、用户认证与权限管理;
  \item 业务层:封装具体业务模块,如用户服务、问诊服务、内容服务等;
  \item 数据层:包括数据库服务、缓存服务(Redis)、文件服务等。
\end{itemize}

\subsection{数据库设计}

平台核心数据通过 MySQL 存储,缓存使用 Redis,搜索引擎使用 Elasticsearch。以“用户表”为例,其逻辑结构如表~\ref{tab:user} 所示。

\begin{table}[H]
\centering
\caption{用户表结构设计}
\label{tab:user}
\begin{tabular}{|c|c|c|c|l|}
\hline
字段名 & 类型 & 长度 & 主键 & 说明 \\
\hline
user\_id & BIGINT & 20 & 是 & 用户唯一标识 \\
username & VARCHAR & 50 & 否 & 用户名 \\
password & VARCHAR & 100 & 否 & 加密后的密码 \\
email & VARCHAR & 100 & 否 & 邮箱地址 \\
created\_at & DATETIME & - & 否 & 注册时间 \\
\hline
\end{tabular}
\end{table}

\subsection{主要模块实现}

\subsubsection{用户管理模块}

该模块提供用户注册、登录、信息修改、身份验证等功能,支持基于 JWT 的 Token 身份认证机制。密码使用 Bcrypt 加密存储。

\subsubsection{在线问诊模块}

该模块实现用户与医生的图文问诊交流,采用 WebSocket 实时通信方式。后端管理用户问诊订单的生成、处理与状态追踪,并支持历史问诊记录查询。

\subsubsection{内容服务模块}

管理员可通过后台发布健康文章资讯,用户可浏览并收藏文章。系统支持关键词搜索,并通过 Elasticsearch 实现模糊匹配和排序推荐。

\subsubsection{问卷量表模块}

支持用户填写心理/健康自评问卷,系统根据用户回答计算分数并给出反馈建议。支持历史评测记录的展示与数据可视化。

\subsection{接口示例设计}

以下为一个用户登录接口的示例说明:

\begin{itemize}
  \item URL:`POST /api/auth/login`
  \item 参数:`username`, `password`
  \item 返回:用户基本信息 + JWT Token
  \item 错误码:401(认证失败)
\end{itemize}

\subsection{本章小结}

本章详细介绍了平台的系统架构、数据库设计、核心模块的功能与实现方法。通过模块化与服务化设计,平台具备良好的扩展性和维护性,为后续部署与测试提供了基础。

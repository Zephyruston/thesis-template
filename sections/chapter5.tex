\section{部署与测试}

本章主要介绍健康管理平台的部署方式、所需软硬件环境及主要功能测试方案,验证系统的稳定性、可靠性与实际使用性能。

\subsection{部署环境}

\subsubsection{软硬件环境}

平台推荐部署于 Linux 服务器(如 Ubuntu 20.04),所需环境如表~\ref{tab:env} 所示:

\begin{table}[H]
\centering
\caption{部署环境配置}
\label{tab:env}
\begin{tabular}{|c|c|}
\hline
组件 & 配置说明 \\
\hline
操作系统 & Ubuntu 20.04 LTS \\
Java 环境 & JDK 17 \\
数据库 & MySQL 8.0 \\
缓存 & Redis 7 \\
容器管理 & Docker 24 + Docker Compose \\
搜索引擎 & Elasticsearch 8.10 \\
\hline
\end{tabular}
\end{table}

\subsubsection{服务部署流程}

1. 使用 Git 克隆后端项目代码;
2. 通过 Docker Compose 启动微服务容器;
3. Nginx 配置反向代理;
4. MySQL 初始化数据库结构;
5. 运行测试脚本,确认服务启动正常。

\subsection{功能测试}

\subsubsection{测试方法}

采用黑盒测试方式,设计测试用例覆盖各模块主要功能点,包括:

\begin{itemize}
  \item 注册登录测试;
  \item 问诊功能流程测试;
  \item 文章推荐与检索测试;
  \item 管理后台功能权限测试;
  \item 系统异常与边界条件处理。
\end{itemize}

\subsubsection{注册功能测试用例(示例)}

\begin{itemize}
  \item 测试编号:U1-1-1;
  \item 输入:合法用户名、密码、邮箱;
  \item 期望结果:注册成功,返回用户信息;
  \item 异常输入测试:用户名为空、密码太短等应返回错误提示;
  \item 实际结果:全部通过,符合预期。
\end{itemize}

\subsection{性能测试}

使用 Apache JMeter 对接口并发访问性能进行压测,结果如下:

- 并发 500 用户时,登录接口平均响应时间:312ms;
- 并发 1000 用户时,系统仍可稳定响应,CPU 占用维持在 70\% 以下。

\subsection{本章小结}

本章对平台的部署流程、运行环境及功能测试过程进行了说明。经测试验证,系统在功能完整性、稳定性与性能表现方面均满足设计预期。

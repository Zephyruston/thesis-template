\section{总结与展望}

本论文围绕“基于微服务的健康管理平台”进行了系统性研究与设计。全文主要完成以下工作:

\begin{itemize}
  \item 分析了微服务架构在健康领域的应用背景与优势;
  \item 基于 Spring Cloud 与 Spring Boot 构建平台整体架构;
  \item 实现了包括用户管理、在线问诊、内容服务、问卷评估等核心功能;
  \item 通过容器化部署、Redis 缓存与 Elasticsearch 提升了系统性能;
  \item 完成平台测试,验证系统具备良好的可用性与扩展性。
\end{itemize}

\subsection{研究成果总结}

本平台结合主流的微服务架构与现代互联网医疗业务需求,搭建了一个功能完善、结构合理、可持续发展的健康管理系统。系统在用户交互体验、模块化开发、服务治理等方面均取得良好效果。

\subsection{不足与未来展望}

尽管平台已具备核心功能,但仍存在以下不足:

\begin{itemize}
  \item 当前问诊服务未对接真实医疗机构,仅做模拟实现;
  \item 缺乏对用户健康数据的深度分析与智能推荐机制;
  \item 移动端支持能力仍需进一步优化。
\end{itemize}

未来可以在以下方向进行扩展:

\begin{itemize}
  \item 对接线下医疗资源,打通服务闭环;
  \item 引入 AI 模型进行健康评估与干预建议;
  \item 优化前端体验,开发小程序等多端支持。
\end{itemize}

\subsection{结语}

本课题作为一项实践与理论相结合的研究工作,探索了微服务在健康平台开发中的可行性,期望能为相关领域的开发者和研究者提供参考与借鉴。
